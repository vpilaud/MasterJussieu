\documentclass[11pt]{article}
\usepackage[a4paper,margin=1.8cm]{geometry}
\usepackage[francais]{babel}
\usepackage[latin1]{inputenc}
\usepackage[T1]{fontenc}
\frenchbsetup{og=�,fg=�}

\title{Proposition de cours au Master Math�matiques et Applications \\ Combinatoire des polytopes}
\author{Arnau Padrol\thanks{Institut de Math�matiques de Jussieu.} \and \addtocounter{footnote}{1} Vincent Pilaud\thanks{Laboratoire d'Informatique de l'�cole Polytechnique.}}
\date{\today}

\begin{document}
\maketitle


%%%%%%%%%%%%%%%%%%%%%%%%%%%%%%%%%%%%%%%%%%%%%%%%%%%%%%%%%%%%%%%%%%%%%%%%%%
\section{Th�me du cours}


%%%%%%%%%%%%%%%%%%%%%%%%%%%%%%%%%%%%%%%%%%%%%%%%%%%%%%%%%%%%%%%%%%%%%%%%%%
\section{Plan du cours}

\begin{description}
\item[1. Cones poly�driques] ~
	\begin{itemize}
	\item D�finition des cones et propri�t�s �l�mentaires
	\item Thm de Minkowski-Weyl pour les cones (preuve par lancer de rayon)
	\item �limination de Fourier-Motzkin
	\item Polarit�
	\end{itemize}
	TD1. 

\medskip
\item[2. Polytopes, poly�dres] ~
	\begin{itemize}
	\item Convexit� -- Thm de Carath�odory -- Thm de Radon
	\item D�finition des polytopes et poly�dres ($V$ et $H$ descriptions)
	\item Exemples (simplexes, cubes, cross-polytopes, $0/1$-polytopes issus de probl�mes d'optimisation combinatoire)
	\item Thm de Minkowski-Weyl pour les polytopes
	\item Polarit�
	\end{itemize}
	TD2.

\medskip
\item[3. Faces] ~
	\begin{itemize}
	\item D�finitions des faces et du $f$-vecteur, propri�t�s �l�mentaires
	\item Treillis des faces d'un poly�dre
	\item Polytopes simples et simpliciaux
	\item Cones et �ventails normaux
	\end{itemize}
	TD3.

\medskip
\item[4. Op�rations et exemples] ~
	\begin{itemize}
	\item Produit Cart�sien
	\item Somme directe
	\item Join
	\item Pyramide
	\item Somme de Minkowski
	\item Projections / sections
	\end{itemize}
	TD4.

\medskip
\item[5. Graphes de polytopes] ~
	\begin{itemize}
	\item Thm de Balinski
	\end{itemize}
	TD5. Thm de Blind-Mani et Kalai (graphes de polytopes simples)

\medskip
\item[6. Relations sur les $f$-vecteurs] ~
	\begin{itemize}
	\item 
	\end{itemize}
	TD6.

\medskip
\item[7. Polytopes extr�maux] ~
	\begin{itemize}
	\item 
	\end{itemize}
	TD7.

\medskip
\item[8. Introduction des matroides orient�s] ~
	\begin{itemize}
	\item 
	\end{itemize}
	TD8.

\medskip
\item[9. Dualit� de Gale] ~
	\begin{itemize}
	\item 
	\end{itemize}
	TD9.

\medskip
\item[10. Espace de r�alisation d'un matroide orient�] ~
	\begin{itemize}
	\item 
	\end{itemize}
	TD10.

\medskip
\item[11. Th�or�mes d'universalit� pour les matroides orient�s] ~
	\begin{itemize}
	\item 
	\end{itemize}
	TD11.

\medskip
\item[12. Th�or�mes d'universalit� pour les polytopes] ~
	\begin{itemize}
	\item 
	\end{itemize}
	TD12.

\end{description}

%%%%%%%%%%%%%%%%%%%%%%%%%%%%%%%%%%%%%%%%%%%%%%%%%%%%%%%%%%%%%%%%%%%%%%%%%%
\section{Liens avec les autres cours}


%%%%%%%%%%%%%%%%%%%%%%%%%%%%%%%%%%%%%%%%%%%%%%%%%%%%%%%%%%%%%%%%%%%%%%%%%%
\section{Organisation}


\end{document}
